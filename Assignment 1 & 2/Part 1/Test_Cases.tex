\documentclass{article}

\title{\textbf{Software Testing Assignment 1}}
\author{\textbf{Alireza Dastmalchi Saei}\\\\
    \textit{\textbf{Student ID:} 993613026}\\\\
    \textit{\textbf{University:} University of Isfahan}\\\\
    \textit{\textbf{Course:} Software Testing Course}\\}
\date{\today}

\begin{document}

\maketitle

\pagebreak

\section{Introduction}


In this comprehensive report, we embark on an exploration and evaluation of a Train Reservation System using a dual-approach testing methodology. Leveraging the power of JUnit and Gherkin, we conduct a thorough examination of the system's functionalities, ensuring its robustness, reliability, and adherence to business requirements.\\

JUnit, a widely adopted testing framework for Java applications, provides a solid foundation for scenario-based testing. Through the creation of five detailed scenarios, we utilize JUnit to meticulously assess various aspects of the Train Reservation System. These scenarios encompass critical functionalities such as City and Train Management, Trip Management, Ticket Booking and Cancellation, Delay Management, and Exchange Management (With Gherkin).\\

Complementing our JUnit tests, we employ Gherkin, a language-agnostic format, to write three additional scenarios. Gherkin facilitates Behavior-Driven Development (BDD) by allowing us to describe system behavior in a natural language format that is easily understandable by both technical and non-technical stakeholders.\\

This integrated approach ensures a comprehensive evaluation of the Train Reservation System, covering both the low-level unit testing provided by JUnit and the high-level behavioral testing facilitated by Gherkin. By combining these two testing methodologies, we aim to provide a robust and thorough validation of the system's functionality, ensuring its readiness for deployment in real-world scenarios.

\pagebreak

\section{Test Scenario 1: City and Train Management}
\bigskip
\bigskip
\subsection{Test Case 1.1: Add a New City}

\textbf{Description:} Verify that a new city can be added to the system.\\
\textbf{Preconditions:}
\begin{itemize}
  \item The system is running.
  \item The user is authorized to add a new city.
\end{itemize}
\textbf{Steps:}
\begin{enumerate}
  \item Add a new city.
\end{enumerate}
\textbf{Expected Result:} The city should be added to the list of cities in the system.

\bigskip
\hrule
\bigskip


\subsection{Test Case 1.2: Add a New Train}

\textbf{Description:} Verify that a new train can be added to the system.\\
\textbf{Preconditions:}
\begin{itemize}
  \item The system is running.
  \item The user is authorized to add a new train.
\end{itemize}
\textbf{Steps:}
\begin{enumerate}
  \item Add a new train.
\end{enumerate}
\textbf{Expected Result:} The new train should be added to the list of trains in the system.

\pagebreak

\section{Test Scenario 2: Trip Management}
\bigskip
\bigskip
\subsection{Test Case 2.1: Create a New Trip}

\textbf{Description:} Verify that a new trip can be added to the system considering the specific constraints.\\
\textbf{Preconditions:}
\begin{itemize}
  \item The system is running.
  \item The user is authorized to add a new trip.
\end{itemize}
\textbf{Steps:}
\begin{enumerate}
  \item Provide valid details for the new trip: Origin
  \item Provide valid details for the new trip: Destination
  \item Provide valid details for the new trip: Train
  \item Provide valid details for the new trip: Departure Time
  \item Provide valid details for the new trip: Arrival Time
\end{enumerate}
\textbf{Expected Result:} The new trip should be created and registered in the system.

\bigskip
\hrule
\bigskip

\subsection{Test Case 2.2: Cancel Trip}

\textbf{Description:} Verify that a trip can be canceled in the system.\\
\textbf{Preconditions:}
\begin{itemize}
  \item The system is running.
  \item The user is authorized to cancel a trip.
\end{itemize}
\textbf{Steps:}
\begin{enumerate}
  \item Select a trip to cancel
\end{enumerate}
\textbf{Expected Result:} The selected trip should be canceled, and all associated tickets should also be canceled.

\bigskip
\hrule
\bigskip

\subsection{Test Case 2.3: Add a trip with conflicting timings to a train}

\textbf{Description:} Verify that a trip with conflicting times cannot be added to the train trips list.\\
\textbf{Preconditions:}
\begin{itemize}
  \item The system is running.
  \item The user is authorized to add a trip.
\end{itemize}
\textbf{Steps:}
\begin{enumerate}
  \item Add a valid trip to train
  \item Add a new trip that has intercepting times with previous trip
\end{enumerate}
\textbf{Expected Result:} The second should not be added to the trips list of the train, and a trip exception must be thrown.

\pagebreak

\section{Test Scenario 3: Ticket Booking and Cancellation}
\bigskip
\bigskip
\subsection{Test Case 3.1: Book a Ticket}

\textbf{Description:} Verify that a new ticket can be booked for a trip if the trip has not reached the maximum number of passengers.\\
\textbf{Preconditions:}
\begin{itemize}
  \item The system is running.
  \item The user is authorized to book a ticket.
  \item There is an available trip.
\end{itemize}
\textbf{Steps:}
\begin{enumerate}
  \item Select and available trip
  \item Provide passenger name
  \item Book a ticket
\end{enumerate}
\textbf{Expected Result:}  A new ticket should be booked for the selected trip.

\bigskip
\hrule
\bigskip

\subsection{Test Case 3.2: Cancel a Ticket}

\textbf{Description:} Verify that a booked ticket can be canceled.\\
\textbf{Preconditions:}
\begin{itemize}
  \item The system is running.
  \item The user has a booked ticket.
\end{itemize}
\textbf{Steps:}
\begin{enumerate}
  \item Select a booked ticket
\end{enumerate}
\textbf{Expected Result:} The selected ticket should be canceled, and the trip should be updated accordingly.

\bigskip
\hrule
\bigskip

\subsection{Test Case 3.3: Book a Ticket for a full trip}

\textbf{Description:} Verify that a ticket for a full trip cannot be created.\\
\textbf{Preconditions:}
\begin{itemize}
  \item The system is running.
\end{itemize}
\textbf{Steps:}
\begin{enumerate}
  \item Create a ticket for a trip with max passengers
\end{enumerate}
\textbf{Expected Result:} The ticket should not be booked and it must give an error.

\pagebreak

\section{Test Scenario 4: Delay Management}
\bigskip
\bigskip
\subsection{Test Case 4.1: Add Departure Delay to a Trip}

\textbf{Description:} Verify that a departure delay can be added to a trip, and it updates the real departure time.\\
\textbf{Preconditions:}
\begin{itemize}
  \item The system is running.
  \item There is a trip available for delay.
\end{itemize}
\textbf{Steps:}
\begin{enumerate}
  \item Select a trip
  \item Add departure delay
\end{enumerate}
\textbf{Expected Result:} The departure delay should be added to the trip, and the real departure time should be updated accordingly.

\bigskip
\hrule
\bigskip

\subsection{Test Case 4.2: Add Arrival Delay to a Trip}

\textbf{Description:} Verify that an arrival delay can be added to a trip, and it updates the real arrival time.\\
\textbf{Preconditions:}
\begin{itemize}
  \item The system is running.
  \item There is a trip available for delay.
\end{itemize}
\textbf{Steps:}
\begin{enumerate}
    \item Select a trip
    \item Add arrival delay
\end{enumerate}
\textbf{Expected Result:} The arrival delay should be added to the trip, and the real arrival time should be updated accordingly.

\bigskip
\hrule
\bigskip

\subsection{Test Case 4.3: Add Departure Delay more than Duration to pass arrival date}

\textbf{Description:} Verify that a departure date cannot pass arrival time after being delayed.\\
\textbf{Preconditions:}
\begin{itemize}
  \item The system is running.
  \item There is a trip available for delay.
\end{itemize}
\textbf{Steps:}
\begin{enumerate}
    \item Select a trip
    \item Add departure delay more that duration
\end{enumerate}
\textbf{Expected Result:} The departure delay should not be added to the trip, or the arrival time must be updated.

\pagebreak

\section{Test Scenario 5: Exchange Management (Gherkin)}
\bigskip
\bigskip
\subsection{Test Case 5.1: X}


\end{document}
